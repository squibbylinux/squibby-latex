%Βασικά στοιχεια δημιουργίας latex
\documentclass[a4paper,12pt]{examsgr}
\usepackage{fontspec,xltxtra,xunicode}
\usepackage{xgreek}

%Γραμματοσειρά κειμένου
\setmainfont{Garamond}

%Πακέτα
\usepackage{enumerate}
\usepackage{enumitem}
\usepackage{multirow}

%Στοιχεία εγγράφου
%=============================================================================
%Όνομα σχολείου
\schoolname{ΓΕΝΙΚΟ ΛΥΚΕΙΟ}
%Τάξη (Α, Β, Γ)
\schoolclass{Γ}
%Ημερομηνία εξέτασης μαθήματος
\schooldate{06-05-12}
%Όνομα μαθήματος
\lesson{ΜΑΘΗΜΑΤΙΚΑ ΚΑΤ.}
%Όνομα διευθυντή
\principal{ΘΕΜΕΛΗΣ ΕΥΡΙΠΙΔΗΣ}
%Ονόματα καθηγητών (χωρίστε τα ονόματα με \\)
\teachername{ΘΕΜΕΛΗΣ ΕΥΡΙΠΙΔΗΣ \\ ΘΕΜΕΛΗΣ ΕΥΡΙΠΙΔΗΣ}
%=============================================================================

\begin{document}
\schoolbasic
%Ξεκινήστε να γράφετε τα θέματα
%=============================================================================
\thema 
Να αποδείξετε το Πυθαγόρειο Θεώρημα και στη συνέχεια να σχεδιάσετε το  αντίστοιχο σχήμα. 

\monades{10}\\

\thema
Να δείξετε ότι
\begin{enumerate}
\item ksdjgbkasjdgb \monades{4}
\item asdljvkakdsjvb  \monades{3}
\end{enumerate}



%=============================================================================
%Σταματήστε να γράφετε τα θέματα
\schoolfinish

\end{document}