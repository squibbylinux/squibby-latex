%Βασικά στοιχεια δημιουργίας latex
\documentclass[a4paper,12pt]{examsgr}
\usepackage{fontspec,xltxtra,xunicode}
\usepackage{xgreek}

%Γραμματοσειρά κειμένου
\setmainfont{Garamond}

%Πακέτα
\usepackage{enumerate}
\renewcommand{\labelenumi}{\roman{enumi}.}
\usepackage{enumitem}
\usepackage{multirow}

%Στοιχεία εγγράφου
%=============================================================================
%Όνομα σχολείου
\schoolname{ΓΕΝΙΚΟ ΛΥΚΕΙΟ}
%Περιοχή σχολείου
\schoolplace{}
%Τάξη (Α, Β, Γ)
\schoolclass{Γ}
%Ημερομηνία εξέτασης μαθήματος
\schooldate{06-05-12}
%Όνομα μαθήματος
\lesson{ΜΑΘΗΜΑΤΙΚΑ ΚΑΤ.}
%Όνομα διευθυντή
\principal{ΘΕΜΕΛΗΣ ΕΥΡΙΠΙΔΗΣ}
%Ονόματα καθηγητών (χωρίστε τα ονόματα με \\)
\teachername{ΘΕΜΕΛΗΣ ΕΥΡΙΠΙΔΗΣ \\ ΘΕΜΕΛΗΣ ΕΥΡΙΠΙΔΗΣ}
%=============================================================================

\begin{document}
\schoolbasic
%Ξεκινήστε να γράφετε τα θέματα
%=============================================================================

%--------------------------------Θέμα Α---------------------------------------
\thema{
\subthema{Να αποδείξετε το Πυθαγόρειο Θεώρημα και στη συνέχεια να σχεδιάσετε το  αντίστοιχο σχήμα. σχεδιάσετε το  αντίστοιχο σχήμα.}{4}

\subthema{Να αποδείξετε το Πυθαγόρειο Θεώρημα και στη συνέχεια να σχεδιάσετε το  αντίστοιχο σχήμα. σχεδιάσετε το  αντίστοιχο σχήμα.}{8}

\subthema{Να αποδείξετε το Πυθαγόρειο Θεώρημα και στη συνέχεια να σχεδιάσετε το  αντίστοιχο σχήμα. σχεδιάσετε το  αντίστοιχο σχήμα.}{8}
}{}

%--------------------------------Θέμα Β---------------------------------------
\thema{
\subthema{Να αποδείξετε το Πυθαγόρειο Θεώρημα και στη συνέχεια να σχεδιάσετε το  αντίστοιχο σχήμα.}{2}
}{}

%--------------------------------Θέμα Γ---------------------------------------
\thema{
Δίνεται ισοσκελές τρίγωνο ΑΒΓ (ΑΒ=ΑΓ) και ΒΕ και ΓΔ οι διχοτόμοι των γωνιών Β και Γ αντίστοιχα. Να αποδείξετε ότι
\begin{enumerate}
\item ryedtgvubhnkjlnmfrygvubyhinjmkl \hfill\monades{2}
\item trbghkmlkdjbgvjdgsvidagsvikvads \hfill\monades{2}
\end{enumerate}
}{}

%--------------------------------Θέμα Δ---------------------------------------
\thema{
Δίνεται ισοσκελές τρίγωνο ΑΒΓ (ΑΒ=ΑΓ) και ΒΕ και ΓΔ οι διχοτόμοι των γωνιών Β και Γ αντίστοιχα.
}{5}

%--------------------------------Θέμα Δ---------------------------------------
\thema{
Δίνεται ισοσκελές τρίγωνο ΑΒΓ (ΑΒ=ΑΓ) και ΒΕ και ΓΔ οι διχοτόμοι των γωνιών Β και Γ αντίστοιχα.
}{5}

%--------------------------------Θέμα Δ---------------------------------------
\thema{
Δίνεται ισοσκελές τρίγωνο ΑΒΓ (ΑΒ=ΑΓ) και ΒΕ και ΓΔ οι διχοτόμοι των γωνιών Β και Γ αντίστοιχα.
}{5}
%=============================================================================
%Σταματήστε να γράφετε τα θέματα
\schoolfinish

\end{document}